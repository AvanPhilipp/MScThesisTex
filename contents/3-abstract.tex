\chapter*{Abstract}
\paragraph{}
The widespread use of unmanned aerial vehicles (UAV) increases the risk of accidents, therefore new procedures must be developed to avoid dangerous approaches and collisions. Due to the size of small UAVs only passive image processing systems can be used. However, high-resolution cameras are required to detect dangerous aircraft from a large distance. High computing performance is required to process real-time image streams arriving from the cameras, which can only be achieved quickly and efficiently by using application-specific image processing architectures.

Such architectures are too large for a UAV both in energy requirement, and size.
One possible solution is to use programmable logic circuits.
The embedded systems are smaller in size and have lower energy consumption compared to the traditional architectures, and its computing performance can be the same due to the application-specific algorithm. 
An embedded system can be also improved by adding other functions, dividing the computational tasks and further extending the possible application areas.

With Programmable Logic, the image processing tasks can be executed on hardware.
The other parts of the algorithm use the integrated processor architecture of the chip.
This approach limits the power consumption even more.

The final product is a hardware design which can process a $3840 \times 2160$ image in 0.04160763 seconds.
This image can either be one UHD camera or can be assembled from 4 different HD camera input.
With 4 wide-angle camera, the whole $360^\circ$ area of the drone can be detected.
The system in the process consumed less than 4 W of energy.

\clearpage
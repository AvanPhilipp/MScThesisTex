\chapter*{Kivonat}
A széleskörben terjedő pilóta nélküli repülőgépek (Unmanned Aerial Vehicles UAV) növelik a a balesetek lehetőségét, ezért elengedhetetlen új módszerek fejlesztése, amik segítenek elkerülni a veszélyes közelségeket illetve baleseteket.
A repülőgépek kicsi mérete miatt csak passzív képfeldolgozó rendszerek telepítése lehetséges, azonban nagyfelbontású kamerák szükségesek egy veszélyes helyzet megfelelő távolból történő érzékelésére.
A nagy felbontású képfolyam valós idejű feldolgozására nagy számítási teljesítmény szükséges.
Ez csak alkalmazás specifikus képfeldolgozó architektúrával lehet elérni.

Ilyen architektúrák azonban mind méretben mind energiaszükségletben túlméretesek a repülőgépekhez képest.
A megoldást a programozható logiaki áramkörök jelenthetik.
Beágyazott rendszerek mind energiafelhasználásban, mind méretben sokkal kisebbek mint a hagyományos architektúrák, ugyanakkor a probléma specifikus algoritmus megvalósításnak köszönhetően a számítási teljesítmény nem csökken.
A beágyazott rendszerek továbbá könnyen kiegészíthetőek más hasonló rendszerekkel, hogy együtt dolgozva a számítási feladatokat megosszák egymással.

Programozható logikai áramkörök segítségével a képfeldolgozó algoritmusokat hardware szinten lehet implementálni.
Az algoritmus többi része a chipen található processzor alrendszeren tud futni.
Ezzel még inkább csökkenthető az energiafelhasználás.

A végeredmény egy olyan hardware design ami képes feldolgozni egy $3840 \times 2160$ felbontású képet 0.04160763 másodperc alatt.
Ez a kép lehet egy UHD kamera képe nagyon részletes felbontással, de állhat 4 különálló HD kamera képéből is.
A négy kamerás megoldásnál a széles látószög lehetővé teszi a drón teljes $360^\circ$ környezetének a megfigyelését.
A rendszer mindezek mellett képes a fogyasztását 4 W alatt tartani.

